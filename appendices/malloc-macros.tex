\section{Constants and macros used to implement \texttt{mm\_malloc}}
\label{app:malloc-macros}

Some of these are based on the course book \cite[p. 893]{computersystems}.

\bgroup
\small
\begin{minted}[bgcolor=LightGray, linenos, breaklines]{c}
/* single word (4) or double word (8) alignment */
#define ALIGNMENT 8
/* rounds up to the nearest multiple of ALIGNMENT */
#define align(size) (((size) + (ALIGNMENT - 1)) & ~0x7)

#define WSIZE 4             /* Word and header/footer size (bytes) */
#define DSIZE 8             /* Double word size (bytes) */
#define CHUNKSIZE (1 << 12) /* Extend heap by this amount (bytes) */

/* Pack a size and allocated bit into a word */
#define make_header(block_size, is_allocated) ((block_size) | (is_allocated))

/* Read and write a word at address p */
#define get_val(ptr) (*(unsigned int *)(ptr))
#define set_val(ptr, val) (*(unsigned int *)(ptr) = (val))

/* Read the size and allocated fields from pointer p */
#define get_size(ptr) (get_val(ptr) & ~0x7)
#define is_allocated(ptr) (get_val(ptr) & 0x1)
#define is_free(ptr) (!(is_allocated(ptr)))

/* Given block ptr bp, compute address of its header and footer. */
#define get_header_ptr(block_ptr) ((char *)(block_ptr)-WSIZE)
#define get_footer_ptr(block_ptr) ((char *)(block_ptr) + get_size(get_header_ptr(block_ptr)) - DSIZE)

/* Given block ptr bp, compute address of next and previous blocks */
#define get_next_block_ptr(block_ptr) ((char *)(block_ptr) + get_size(get_header_ptr(block_ptr)))
#define get_prev_block_ptr(block_ptr) ((char *)(block_ptr)-get_size(((char *)(block_ptr)-DSIZE)))

// Given a block pointer to a free block, get the free block it points to as the next one
#define get_next_free_block_ptr(block_ptr) ((void *)get_val(block_ptr))
// Given a block pointer to a free block, get the free block it points to as the previous one
#define get_prev_free_block_ptr(block_ptr) ((void *)get_val((char *)(block_ptr) + WSIZE))

/*
 * Given the size of the data, calculates how many bytes are needed to store the
 * block:
 *   - 1 word for header
 *   - 1 word for footer
 *   - 8-byte aligned data size
 */
#define calc_block_size(data_size) (WSIZE + WSIZE + align(data_size));

#define max(x, y) ((x) > (y) ? (x) : (y))
#define min(x, y) ((x) < (y) ? (x) : (y))

static int MIN_BLOCK_SIZE = calc_block_size(1);
\end{minted}
\egroup
