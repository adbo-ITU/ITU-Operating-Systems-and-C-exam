\subsection{Exceptions: traps, faults, and aborts}

The very short version of the difference between traps, faults, and aborts is \cite[p. 762]{computersystems}:

\begin{description}[labelindent=1cm]
  \item[Traps] are expected by the programmer
  \item[Faults] are not expected (an error), but they can potentially be recovered from
  \item[Aborts] are not expected, and you cannot recover from them
\end{description}

All three are examples of synchronous exceptions, meaning they arise directly as the result of executing some instruction. What happens after the exception depends on the type of exception. After a trap handler finishes, control is returned to the next instruction in the program. When a fault occurs, control is transferred to a fault handler~\cite[765]{computersystems}. If the handler can successfully handle the error, the instruction that caused the fault is re-executed. Otherwise, the program is aborted. Aborts always cause the program to be aborted.

As an example, traps are used when making system calls. That is, request that the kernel perform some action and return to the running program afterwards. An example of a fault is a page fault, which happens when a physical page of memory is requested, but that page has not been mapped in virtual memory. This should be one that can be recovered from. Aborts can occur for \textit{many} reasons: division with 0, attempting to access memory outside the program, hardware errors, and many more. Aborts often surface as a ``general protection fault'' (typically presented as a ``segmentation fault'') \cite[p. 765]{computersystems}.

One extra note: an interrupt is another type of exception. It is an \textit{asynchronous} exception, caused by a signal being sent from an I/O device~\cite[p. 762]{computersystems}. For example, a network device may want to state that it has received a new connection request. Interrupts, like traps, also return control to the next instruction in the program~\cite[p. 763]{computersystems}.
