\subsection{Memory leaks}

The term \textit{memory leak} refers to when a program heap-allocates memory and does not free it. When this is done repeatedly, the program will continuously use more and more memory until there is none left.

It's worth noting that any memory used by a process will be freed once it terminates. Therefore, memory leaks will typically hit long-running programs harder than programs that are run quickly many times.

An example of a function that causes a memory leak is this one:

\begin{minted}[linenos, bgcolor=LightGray]{c}
char read_first_letter() {
  char *name = (char *) malloc(128);
  scanf("%s", name);
  return name[0];
}
\end{minted}

The function simply reads a string from stdin and returns the first character of that string. On line 2, a 128-byte buffer to hold the name is allocated on the heap, but \code{free(name)} is never called. Thus, every time \code{read_first_letter()} is read, a new buffer is allocated and never freed. Over time, if the function is called many times, it may cause issues.

The best way to avoid memory leaks is by cleaning up all heap-allocated resources that have been created with \code{malloc} with a corresponding call to \code{free}.

You can also choose to implement a garbage collector that automatically detects unused memory and frees it. But if you reach that point, perhaps you should reconsider your choice of using C for the task at hand.

In other programming languages that use a garbage collector, memory leaks can be caused by objects that point to each other alongside other reasons. However, I have not covered that here since the C is the scope.
